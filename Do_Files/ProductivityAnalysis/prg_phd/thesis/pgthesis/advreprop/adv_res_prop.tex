% ch1.tex

%%%%%%%%%%%%%%%%%%%%%%%%%%%%%%%%%%%%%%%%%%%%%%%%
%Comment out this section and \end{document} at the bottom before running thesis.tex
%%%%%%%%%%%%%%%%%%%%%%%%%%%%%%%%%%%%%%%%%%%%%%%

%Choose one of the following and comment the other out

\documentclass[12pt]{report}
%\documentclass[11pt,twoside,final]{huthesis2}
%\documentclass[12pt,oneside,final]{huthesis2}

%\def\dsp{\def\baselinestretch{2.0}\large\normalsize}
%\def\ssp{\def\baselinestretch{1.0}\large\normalsize}
%\dsp

\usepackage{epsfig,bm,epsf,float}
\usepackage[colorlinks=true, citecolor=black, bookmarks, bookmarksopen]{hyperref}
\usepackage{natbib} %[longnamesfirst] for citing all authors' names on first reference
\usepackage{har2nat}
\usepackage{eurosym}
\usepackage{enumitem}
\usepackage{amsmath}

\title{Dairy Economics, Expansion and Sustainable Grassland: \emph{Advanced Research Proposal}}
\author{Patrick R. Gillespie}
\date{13 April, 2011}

\begin{document}
\maketitle
\begin{abstract}
The dairy industry is one of the most important sectors of Irish Agriculture accounting for 30 percent of output. The thesis will draw on newly available data from the National Farm Survey (NFS) from the pre-quota period in Ireland. The goals are: to better understand the relative benefits of the grass based-system, to compare the productivity of the grass-based Irish dairy system with productivity in other countries and to assess the impact of the introduction of the quota regime on dairy farm behaviour in order to draw lessons for the future.
\end{abstract}

%\input{mathdefs} % my math definitions.
%\dsp

%%%%%%%%%%%%%%%%%%%%%%%%%%%%%%%%%%%%%%%%%%%%%%%
\setcitestyle{author-year,round}
\chapter{Introduction}
%%%%%%%%%%%%%%%%%%%%%%%%%%%%%%%%%%%%%%%%%%%%%%%%

\section{Motivation}

%To be changed**From eucom09.pdf**************************************************

%The EU milk quota system was originally introduced in 1984, in order to limit public expenditure on the sector, to control milk production, and to stabilize milk prices and the agricultural income of milk producers. 
%
%%Since the milk quota regime was introduced, milk quota has become a scarce production factor: on the one hand limiting milk production and, on the other hand, stabilising milk producer prices and maintaining dairy activities in less competitive regions. However, in the course of time European dairy policy has been continuously changing and has increasingly encouraged producers to be more market-oriented. Policy developments, including reductions of intervention prices and specific quota increases of various amounts to MS, together with most recent market developments, have provoked that quota is no more binding in some MS and regions of the EU. 
%
%%With the Luxembourg Agreement on the Mid-Term-Review (MTR) on 26 June 2003, the spotlight shifted again on the EU's milk quota regime, because
% the MTR stipulated that the milk quota system will come to an end in 2015. 
%
%Within the Health Check of the Common Agricultural Policy (CAP) the European Commission endorsed the proposal of milk quota abolition and suggested an increase of quota by $1$ percent annually from 2009 to 2013 to allow a "soft landing" of the milk sector to the end of quotas. In this context it is especially important to clarify, which economic effects can be expected of an abolition of the milk quota regime.
%


%************************************************************************

%The Milk Quota regime is now set to come to an end in 2015 after a series of gradual expansions are implemented in order to engineer a `soft landing' for the sector~\citep{eucom09}. The regime has been in place in the European Unioni~(EU) since 1984, and its abolition is a major sea change in policy.   

The dairy industry is one of the most important sectors of Irish Agriculture accounting for 30 percent of output. The sector makes a major contribution to the Irish economy in that it employs approximately 20,000 dairy farmers, 9.000 employees in the processing industry and supporting an additional 4,500 in ancillary services. It also makes a significant contribution to sustaining rural communities. %One of the ten priority actions identified in Teagasc Foresight 2030 report was the need for Teagasc to provide a leadership role in the context of an expansion in milk production. 
It is now accepted that the EU Commission intention is not to extend the dairy quota regime beyond 31 March, 2015. The outcome of the ‘Health Check’ is now known allowing for a 1 percent increase in milk quota each year for the next five years (2009/10 to 2013/14) which is additional to the once off 2 percent increase agreed in 2008/09~\citep{eucom09}. Additional to this it was agreed to reduce the butterfat adjustment coefficient from 0.18 to 0.09 where fat content exceed the reference level which will potentially increase Irish milk quota by a further 2 percent. 

%The project will help to
The goals of this thesis are:
\begin{itemize}
\item  to better understand the relative benefits of the grass-based system compared with other input systems,
\item  to compare the productivity of the grass-based Irish dairy system with productivity in other countries,
\item  to use newly available data on farm management practices on dairy farms for the purpose of assessing their relative contribution to productivity and
\item  to assess the impact the introduction of the quota regime had on dairy farm behaviour in order to draw lessons for the future. 
\end{itemize}

%The relative cost of grass production versus the use of other feed stuffs relates to a number of factors including the cost of fertiliser, the opportunity cost of grass maintenance and the cost of other feed stuffs. The demand for grass and the effort and resources put into the maintenance of grassland will depend in part upon the relative cost of alternative feeds and input costs. The shape of the demand curve for grass will be estimated as a function of the price of grass as defined by the cost of inputs and the opportunity cost of time and the relative cost of alternative feedstuffs. These data are contained within the National Farm Data data, which exists for different enterprise farms and thus allow for the price information to be identified within the demand system. The elasticity can be estimated using a demand system ~\citep{deaton80}. The objective of this work package will be to estimate a demand system for animal feed on the farm. This model will allow us to understand the relative demand for feed under alternative economic scenarios.

%Utilising this demand system, we can also measure the welfare gain under different scenarios of the economic benefit of grass production. In this study we employ our micro-economic framework to assess the welfare gain of using grass versus other food stuffs. Here we utilise a concept known as consumer surplus, which is the difference between the maximum total price a consumer would be willing to pay for the amount bought and the actual total price. The consumer surplus can be defined as the area under the demand curve above the actual price paid. The resulting model can then be used to simulate the relative benefit of grass under alternative price scenarios for inputs and alternatives and aggregating individual farm consumer surplus, generate national added welfare from the advantages due to grass production. This paper will help to underpin the economics of grassland research aimed at improving the knowledge base of grassland technologies.

\subsection{The Demand for Grass Feed}
The relative cost of grass production versus the use of other feed stuffs relates to a number of factors including the cost of fertiliser, the opportunity cost of grass maintenance and the cost of other feed stuffs. The demand for grass and the effort and resources put into the maintenance of grassland will depend in part upon the relative cost of alternative feeds and input costs. The shape of the demand curve for grass will be estimated as a function of the price of grass as defined by the cost of inputs, the opportunity cost of time and the relative cost of alternative feedstuffs. These data are contained within the National Farm Data data, which exists for different enterprise farms and thus allow for the price information to be identified within the demand system. The elasticity can be estimated using a demand system - see for example~\citep{deaton80}. The objective of this work package will be to estimate a demand system for animal feed on the farm. This model will allow us to understand the relative demand for feed under alternative economic scenarios.

Utilising this demand system, one can also measure the welfare gain under different scenarios of the economic benefit of grass production. A micro-economic framework is employed to assess the welfare gain of using grass versus other food stuffs. The applicable concept  is that of consumer surplus, which is the difference between the maximum total price a consumer would be willing to pay for the amount bought and the actual total price. The consumer surplus can be defined as the area under the demand curve above the actual price paid. The resulting model can then be used to simulate the relative benefit of grass under alternative price scenarios for inputs and alternatives and aggregating individual farm consumer surplus, generate national added welfare from the advantages due to grass production. This paper will help to underpin the economics of grassland research aimed at improving the knowledge base of grassland technologies.

\subsection{A Comparative Analysis of the Productivity of Dairy Farms in the EU}
The objective of this task is to examine the factors that influence the level of technical  efficiency at farm level. If such factors could be accurately identified and rectified, farming efficiency, and thus profitability, could potentially be improved. This task proposes to employ the model proposed by \citet{orea04} to find the determinants of technical efficiency. This model will be adapted to capture the independent effects of different grazing intensities on technical efficiency.  The overriding hypothesis of this task is that technical efficiency will differ depending on grazing intensity levels.  In order to estimate the efficiency model a second set of independent variables are required and are assumed to affect the efficiency at which farms convert their factors of production into output. Theory does not point to any specific variables that should be included – it is more of an empirical question.  As such, variables are selected on the basis of economic intuition.  Existing variables might include: the farm’s soil quality, the use of the extension service, the presence of an off-farm job, the use of artificial insemination, the overall farm size, the age of the farmer and the degree of specialisation and  intensification. Additional variables capturing the grazing system could also be incorporated, such as level of grazing intensification. It is proposed to expand the Irish model~\citep{carroll08} to include additional grazing system variables and replicate the Irish model using data from additional EU countries.  

\footnotetext {The model proposed by \citet{orea04} assumes a time-varying inefficiency term ($u_{it}$) as the product of an exponential function of time-varying efficiency variables ($z_{it}$) and a nonnegative, time-invariant firm-specific inefficiency term ($u_i$ ):
\[
	u_{it} = u_i * \exp(z_{it} \delta)
\]
where $\delta$ are parameters to be estimated. A form of this model has been employed by \citet{kumbhakar08a} for the Spanish Dairy sector and \citet{carroll08} for the Irish dairy, beef, sheep and cereals sector.
}


\subsection{Dairy Farm Management Practices and \newline Productivity}
This task will draw upon recent data collected by the National Farm Survey in relation to the uptake of Moorepark dairy technologies and management practices. These include calving dates, grassland management, the use of clover, the use of low cost housing and participation in dairy discussion groups. This research willl extend the Orea and Kumbhakar model in developed in task 2 for Ireland to understand the relationship between adoption of good practice to quantify the impact on dairy farm productivity in Ireland.

\subsection{Understanding Farm Level Behaviour without \mbox{Milk Quotas}}
 While Teagasc has excellent data in the National Farm Survey on dairy farm characteristics, at present in the context at looking at the potential for expansion, it is limited in that the only data currently available for analysis (1984-2008) is from the period where production was limited by milk quota. At present data from the 1975-1983 is being recaptured from historic magnetic tapes and will be available for analysis in 2011. It is hoped to use this data within this project to look at decisions that were made on dairy farms during the expansion phase prior to 1984.

While initially the paper will be descriptive in nature, tracking changes in farm management and investment decisions, the intention is estimate a series of econometric models to compare these decisions with and without quota constraints.

\section{Survey of the literature}

\subsection{The quota regime and competitiveness in the dairy sector}
Much effort is now being put into understanding the competitiveness of dairy sectors in the various  European Union (EU) member states in anticipation of the abolition of milk quota. A dataset which is often employed for this purpose is the  Farm Accountancy Data Network (FADN). The FADN dataset is an unbalanced panel containing farm level accountancy data including output, income, costs, investment, and debt across all EU member states. \citet{donnellan09} use  FADN data on the years 2001 to 2003 and commodity price projections supplied from the Food and Agricultural Policy Research Institute (FAPRI) to analyse international competitveness across EU member states in terms of cash and imputed costs and to draw conclusions about the future state of dairying in the EU. They suggest that member states which implemented the milk quota system in a freer market structure are better prepared for the challenges ahead. Ireland is not considered to have done this, but the authors also imply that Ireland's pasture based systems will be a competitive advantage if high input prices prevail due to, for example,  continued expansion of biofuel production. Even so, they say, it is hard to envisage EU production outputs levels being maintained in a more liberalised dairy market unless the sector undergoes a dramatic structural change.

The \citet{eucom09} used a partial equilibrium modelling framework to examine the likely impact of milk quota abolition. The reports conclusions include a projection that Ireland will dramatically increase its milk production after abolition of the milk quota system and that this will be accompanied by an increase in herd size and a simultaneous \emph{decrease}
on the order of 4.5 percent in agricultural income. \citet{bouamra08}, using a spatial equilibrium model,  find that by 2014-2015, the market effects of abolishing quota are not very different from those of a 2 percent gradual increase starting in 2009. \citet{bouamra00} had argued public welfare grounds that reducing export refunds would be preferrable when compared to milk quota expansion in an earlier analysis using a short-run partial equilibrium model. 

\citet{hennessy09} construct an optimisation model to estimate the cost of inefficiencies created by a regionalised quota system. They solved their model using data from the National Farm Survey (NFS), which is the Irish contribution to the FADN, and conclude that the cost is approximately \euro27 million.

\citet{colman05} use a pooled-OLS and a fixed effects model to examine the short-run own-price elasticities of panels of farms in the Northwest and the Southwest of England. They find evidence that English farmers, operating in a freer market for milk quota, have an own price supply elasticity of 0.27 to 0.29 in the Northwest and 0.33 to 0.36 in the Southwest.

In an earlier study of the canadian dairy sector, \citet{richards96} found that the existence of a quota system created overinvestment in cattle and family labour, which slows total factor productivity growth and competitiveness. Interestingly, he also finds that simulations under different policy regimes showed that total factor productivity growth was 4 percent lower with quotas relative to a policy which supports the milk price, but does not introduce a quota license requirement. He also concludes that creating a national quota system for Canada (i.e. making the quota transferable across regions) has no effect with regard to total factor productivity. 

\citet{kumbhakar08} used panel data of Norwegian dairy farms over the time period 1976 to 2005 to examine the effects of the milk quota regime on the rate of output growth. The authors specified a growth rate model to control for farm-specific effects and to help with theoretical consistency (i.e. non-negativity constraints on input elasticities). The time period covers three quota schemes: pre-quota; restrictive quota implementation; and a flexible quota scheme. They conclude that the quota system slowed output growth and technical change, and that these measures would improved if the quota regime were liberalised.

\citet{flaten03} used a linear programming model to examine the effects of differing assumptions regarding price, headage payments, and quota on the usage of forage crops, herd size, and yield per cow.

\citet{jesse07} conducted a case study of the Irish dairy sector. Key points included: Ireland's climate and geographic location make it ideally suited for a grass-based production system; the milk quota regime has had a negative effect on output growth and structural change in the sector; and much of the change which will occur post-quota abolition will be positive, with some estimates of output growth at around 20 percent.

\citet{wieck07} use the FADN data on the years 1989 to 2000 to construct a model which estimates multi-input multi-output Symmetric Generalized McFadden cost functions. Their analysis addresses hypotheses of the nature of marginal costs in the dairy sector and their relationship to aspects of the enterprise such as farm size, factor endowment, yields, proportion of grassland, degree of specialization, and so forth. They take the analysis beyond the determinants of marginal costs to also examine differentiation in marginal costs, and the development of those costs over time. They conclude that milk output, milk yield, degree of specialization, and farm size all have negative relationship to marginal costs, but grassland shares are positively correlated across regions. It should be noted that the authors used grassland share as a proxy for spatially remote geographic locations, and that they do not imply any relationship between marginal costs and choice of input system.  They also found evidence that marginal costs decrease over time, but limitations in their model prevented them from tying this to differences in implementation of the milk quota regime. The authors identify avenues for further research which include the specification of more sophisticated models  and using datasets which have longer time horizons to investigate determinants of structural change. 

A sufficient understanding of technical efficiency is vital to the study of relative competitiveness. Measures of technical efficiency must be ``theoretically valid and subject to unambiguous interpretations''~\citep{tsionas04}. The reality of technical efficiency is that, despite the assumptions of the neoclassical paradigm, many countries produce below the production possibility frontier for their set of inputs. Many researchers adopt the stochastic frontier~(SF) family of models in order to accomodate both this technical inefficiency and noise in the data. Tsionas and Kumbhakar choose a SF model and combine it with a Markov switching framework to allow for changes in the technical efficiency (i.e. the production frontier). The authors assert that this approach  is useful for cross country comparisons, as technical efficiency growth may well differ between countries and over time.

\citet{kumbhakar08a} investigated the possibility that estimates of measures such as returns to scale, partial production elasticities, technical efficiency, and so forth differ based on the choice of specifying an input-oriented (IO) or output-oriented (OO) model. They applied their analysis to Spanish dairy farms and found that OO modesl predict greater benefits from scale economies than do IO models. Similarly, estimates of technical efficiency had a higher mean and dispersion in the OO models. Finally, they use a Monte Carlo simulation to show that when the returns to scale are close to unity, then the differences between the models estimates are small. 

\citet{conradie09} uses South African agricultural census data from 1952 to 2002 to examine total factor productivity growth, and also to extoll the benefits of disaggregating the data below the national scale to reveal regional trends.

\citet{paul04} use stochastic production frontier (SPF) and deterministic data envelopment analysis (DEA) models to compare the competitiveness of small family farms relative to large industrial farms using a panel of farms in corn belt of the U.S.A. Competitiveness is measured in terms of scale economies and technical efficiency. Their analysis concludes that small family farms are both scale inefficient and technically inefficient.

\citet{nehring09} researched the competitiveness of small farms in the U.S.A. using panel data ranging from 2003 to 2007. They also used a SPF model to compare the performance over time of conventional and pasture technologies which they identified using a binomial logit model. They too found that large conventional farms won out in most economic measures. 

\citet{hailu05} used two non-homothetic translog stochastic meta-frontier cost function to estimate cost and input demand functions for pooled data from Alberta and Ontario provinces in Canada. The average cost efficiency was approximately 89\%, and so the authors conclude that the sector has scope to improve. They also examined the relative competitiveness between the two regions, but were unable to draw conclusions regarding this comparison as the data did not provide any statistical evidence of any difference. 

\citet{douc00} used a SF model to analyse the total factor productivity of the Australian dairy processing sector. They find that a recent trend towards deregulation of the industry there coincides with a slowdown in productivity growth and technical progress. An application of data envelopment analysis (DEA) for dairy farms in Western Victoria, Australia is given by \citet{carter10}. 

\citet{abdulai07} used a `true' random effects model to estimate the technical efficiency of individual farms in northern Germany, as well as estimating efficiency with more conventional panel data models. They also extended the analysis to explore the effects of different specification of the production function on efficiency estimates.

\citet{maietta00} opted for a shadow price approach to measuring inefficiency. Maietta decomposed inefficiency into technical and allocative inefficiency, allowing the former to be modelled in a fixed effect term but keeping the allocative inefficiency through input specific parameters that scale market prices. The model is then applied to panel data of dairy farms in Northern Italy, with the resulting estimate of cost excess due to inefficiency at 69 percent. Most of this inefficiency is due to technical inefficiency, and of the allocative inefficiency, most is due to under-utilisation of forage crops and of purchased feeds with respect to hired labour.

\citet{gillespie09} used a multinomial logit model to compare the profitability of U.S. dairy farms based on production systems in three classes: pasture-based; semi-pasture-based; and conventional. Whilst they found that region, farm size, and demographics affect profitability, and they also found that a conventional system was more profitable than a semi-pasture-based system. However, they were unable to find evidence of an effect on profitability with respect to the choice of a pasture-based system as opposed to the other systems.

\subsection{Dairying and the environment}
Dairy farms' interaction with the environment is also a topic of concern in the literature. One could use a mathematical approach to elucidate the trade\nolinebreak-off between economic and environmental goals, as did \citet{vandeven07} when they used a multiple goal linear programming model. Their model identified farm systems which are technically feasible under environmental and economics constraints, and their model's output ``represents an end of development path with regard to goals pursued and restrictions imposed '', but they advocate the use of a model which incorporates farm households and markets for a short term analysis. 

The new emphasis on grass-based production has roots in both the commercial and environmental spheres. On the one hand grass-based systems may introduce a lower cost structure for production in a country such as Ireland, with its competitive advantage in grass production both in the present~\citep{jesse07} and in the likely future climate~\citep{fitzgerald09}. This would improve the competiteness of the Irish dairy sector. On the other hand choosing a grass-based system may have beneficial environmental effects under the right conditions and management~\citep{kristensen05}.If targets for agricultural emissions are set and enforced equally across the EU, or if consumers place a sufficiently high value on more environmentally friendly dairy products, then Ireland could find that this also increases its competitiveness.  

\citet{donnellan09a} use a partial equilibrium commodity model combined with a GHG calculation model to anticipate the scale of adjustment that would be necessary to meet hypothetical abatement targets in agriculture. That model suggested that an anticipated expansion in the dairy herd in a post-quota world would have to be accomodated by reductions in the suckler herd at a ratio which is greater than one-to-one if the targets are to be met. This analysis used the Intergovernmental Panel on Climate Change's (IPCC) accounting system for GHG emissions in order to remain consistent with government reporting, and this result depends on that choice. As pointed out in a report from the U.S. Department of Energy~\citep{doe99}, efficiency gains in yield per cow can reduce the amount of methane per unit of milk. 

\cite{schneider02} apply mathematical programming to the question of potential GHG mitigation from the U.S. agricultural and forestry sectors and find that carbon prices affect the choice of strategy for mitigating carbon. Lower prices encourage strategies to change soil and livestock practices, while higher prices promote afforestation and biofuel production. A linear programming model is applied to the broader theme of sustainable farming for Dutch dairy farms in \citet{vancalker04}, whilst \citet{brown05} describe a decision tool for grassland management with the goal of increasing N-efficiency.

\citet{thornton10} attempt to quantify the GHG emissions abated through the adoption of improved grasslands in the tropics using a partial equilibrium agricultural commodity model combined with a GHG conversion model. At \$20 per tCO$_2$-e, they estimate that the value of abated and sequestered emissions under their ``optimistic but plausible'' scenario could be worth \$1.3 billion.

\citet{saunders07} compare the U.K. and New Zealand dairy sectors in terms of energy use and GHG emissions on a per unit of output using a life cycle assessment methodology. They find that New Zealand enjoys an advantage; milk solids produced in the U.K. had 34 percent more emissions per kg  (or 30 percent more per ha) than those produced in New Zealand even after accounting for the additional energy and GHG emissions associated with transport. This was a reflection of the less intensive production systems employed in New Zealand.

\citet{reinhard99} use an SF model, specifically a stochastic translog production frontier, to estimate both the technical and environmental efficiency of a panel of Dutch dairy farms. They measure output oriented (OO) technical efficiency and input-oriented (IO) environmental efficiency where the input is a single detrimental environmental input. They find that intensive farms are both technically and environmentally more efficient than extensive farms. \citet{reinhard00} conduct a similar analysis, but one which focuses on an input-oriented measure of both technical and environmental efficiency (N surplus). They find that the N surplus at the N-efficient point is less than half of the surplus observed in the data.

\citet{nevens06} combine data from discussion groups, experimental farms, a literature review, and data from the FADN for an analysis of changes in N surpluses on Dutch farms. They find that Dutch farms have become much more N efficient, but that there is still scope for vast improvement, as evidenced by the most progressive farms in the Netherlands. 

\citet{groot06} use a policy change in the Netherlands as a natural experiment by which they may analyse the effects of differing strategies for improving nitrogen use efficiency. They find that three groups of farms were most successful in implementing the change. Those that implement a consistent strategy (usually typified by continuous gradual adjustment of integrated farm management combined with varying strategies regarding productivity) realised a higher margin than those that did not. However, no relationship between improved nitrogen use efficiency and gross margin could be shown. The authors' speculate that the choice of adoption of a particular strategy may be decided by factor endowments, farmer's skills, and the like.

Like \citeauthor{vandeven07}, \citet{thomassen09} explore the trade-offs between economic and environmental goals for dairy farms. They also used FADN data, but their study focused on a cross section of Dutch farms in 2005. They identified several environmental indicators based on life cycle assessment of output (i.e. they give measures on a per unit of milk basis). A key conclusion from that study was that environmental indicators were mainly affected by milk production per hectare, milk production per cow, farm size, and the amount of concentrates per kg of milk. They also found that some environmental goals are competing (e.g. an increased milk yield per cow is associated with lower GHG emissions per unit of milk and decreased soil quality per hectare at a constant stocking density). 

Farm interactions with the environment are a complex system of cycling of materials on the farm and exchange with the environment in which the farm is situated. Acknowledging this, \citet{schils07} conduct a survey of whole-farm models to assess their suitability for understanding this system, and they conclude that they are indeed powerful tools to use in this regard. The authors also point out the possibility to use farm gate N surplus as a proxy for GHG emissions, given the correlation between the two variables.

\subsection{The link to biological factors and the capacity for adaptation}
The tie to the biological sphere is of greater importance in agricultural production than it is in most other forms of modern production. This makes bio-economic models very relevant for a studies of efficiency in the sector.~\citet{tozer99} apply such a model to the Australian dairy industry. They find that low-input systems are most constrained by grass feed availability. 

Competitiveness relies on technical efficiency, but also on technical development. This development often takes the form of improved genetics in the dairy industry. The use of Artificial Insemination (AI) is one avenue for pursuing advancements in this area, and it has been shown to be more profitable than Natural Service (NS) under various environmental and management scenarios~\citep{valergakis06}.  

\citet{ledgard99} used a 3 year study on experimental dairy farms in New Zealand to examine the effect of differing levels of nitrogen application on mixed clover and grass pastures to N-efficiency. He found that a pasture system whose sole reliance on N$_2$-fixation as its main source of N was relatively efficient at 52\% of N output in products relative to N input. They also found that on a 400 N farmlet, the N-efficiency was still favourable as compared to intensive dairy systems in England and the Netherlands. \citet{ledgard09} found that at similar inputs, mixed clover and grass pastures can be more N-efficient than grass only pastures, but they also stipulate  that ``other management practices on the farm [...] can have a larger overall effect on environmental emissions than whether the N input is derived from fertiliser N of from N$_{2}$ fixation.'' 

\citet{shalloo04a} use data from a three year study to construct a bio-economic model to investigate the genetic potential for milk production under various levels of concentrate supplements in the diet. They then applied differing assumptions regarding price levels and quota regime in a stochastic budgeting framework. The results indicated that when quota was tied to individual farms, the optimal system is where margin per unit of output is maximised. The optimal system changes to one where margin per cow is maximised when quota is transferrable. They also imply that the optimum system for breeds with low genetic potential is a low concentrate system, while high concentrates are optimal for high genetic potential breeds. 

\citet{shalloo04} also construct a model which examines the productivity of pasture based spring calving dairy systems on high rainfall, heavy clay soil versus those which operate on lower rainfall, free draining soil. They then quantify the difference in their economic potentials applying a Monte Carlo simulation approach to the observed data. They found that at a 468,100 kg EU quota scenario, profitability on the lower-rainfall free draining soil farm was \euro27,417 when differences in yield per cow were allowed. When those yields were held constant for both farms the difference in profitability was still \euro19,138 in favour of the well drained soil. The authors draw the conclusion that production may not be viable on heavy clay soil types in Ireland in the future. 

\citet{kristensen05} survey the literature to examine the link between grassland management practices, environmental and production outcomes. They call for more research in the areas ranging from nutrient utilisation, to the interaction of N and carbohydrates in dairy cattle nutrition, to ``improved knowledge of how to manage grassland taking into account herd size and the interactions between technology and grassland management''. 

\citet{hopkins07} reviewed the literature regarding the potential effects of climate change on grassland yields. They found that there is a potential for increase herbage growth and for the new conditions to encourage legumes more than grasses. However, grassland agriculture can also contribute to GHG emissions, and as a result more effort has to go into understanding management practices  which mitigate these emission ``in  a holistic way that also considers other pressures''. In the same vein, \citet{fitzgerald09} utilise a simulator to anticipate the impacts of climate change on low-cost, grass-based dairying in Ireland. The simulation was carried out for 11 locations for both well drained and poorly drained soil types, and the changes in system properties (e.g. stocking rate, grass yield, etc.) over the simulation period were observed. The authors conclude that Ireland's grass-based dairying systems should be able to adapt to the future climate well. In a similar study carried out for the Swiss plateau, \citet{finger10} find that climate change increases risks in grassland production. 

\citet{herrero99} construct a bio-economic model using simulation and mathematical programming techniques to represent grass-based dairy systems. The resulting model is applied to highland dairy perfomance in Costa Rica, and compares observed data to the optimal outcome detailed by the model. The model's attention to pasture based systems is pertinent, as is its attention to regional context in the biological aspects. 

\chapter{Methodology}
\section{Introduction}
\subsection{Policy Background}
The dairy industry is one of the most important sectors of Irish Agriculture accounting for 30 percent of output. It makes a major contribution to the Irish economy in that it employs approximately 20,000 dairy farmers, 9.000 employees in the processing industry, and supports an additional 4,500 in ancillary services. It also makes a significant contribution to sustaining rural communities. Within a policy context of the impending abolition of Milk Quota and the strategic goals set forth in the Harvest 2020 programme, it is of interest to investigate both the drivers and the barriers to Total Factor Productivity (TFP) growth in the Irish dairy sector.

The Milk Quota regime was established in 1984 under Reg. 856/84 . It was created as a response to the chronic over supply of milk products in the EU, which was itself the result of the various price protections that are set forth in the Common Agricultural Policy (CAP)\footnote{See \citet{fennell87}} . The quota system sought to remedy this by limiting individual producers in their right to supply product for the market. A superlevy is imposed if an individual firm produces in excess of its quota, and this generally makes the excess production uneconomic. 

The outcome of the round of reforms of the CAP known as the `Health Check' (Reg. 72/2009) allowed for  a 1 percent increase in milk quota each year for the next five year period starting in 2009 and running through 2014. This is in addition to the once off 2 percent increase agreed in 2008/09. It is now accepted that the EU Commission intention is not to extend the dairy quota regime beyond 31 March, 2015. 

\subsection{Hypotheses}
The interaction between the Milk Quota regime and TFP is of particular interest now that the system is set to expire. Improvements in TFP can be understood as increasing the amount of output for a given set of inputs. When one thinks of TFP in this way one can easily see how a quota system might reduce the incentive to innovate and thus negatively affect TFP growth. However, the restricting of supply in and of itself does not imply negative effects on Total Factor Productivity, as improving TFP can equally be thought of as reducing the amount of inputs required for a given level of output.  This level of input may simply be set at the level of quota allowance for any firm. 

However, the Milk Quota system differs in application from country to country. Some countries allow for the trading of quota at the national level, but Ireland implements a system whereby quota is ring-fenced in geographical regions. Using an optimisation model \citet{hennessy09} found that a regionalised quota system creates \euro27 million in economic inefficiency at the national level. One might also imagine that the costs of such a system as implemented in Ireland may not only have been felt as an inefficient allocation of resources, but also in terms of foregone improvements in technical efficiency. In other words, the quota system may have stymied TFP growth. 

	\begin{itemize}[label=]
		\item{\textbf{$H_{1}$: The establishment of the milk quota regime had a \\ negative effect on total factor productivity in Ireland.}}
	\end{itemize}

With the removal of Milk Quota, Ireland will be free to increase dairy output, as Ireland has been shown to be highly competitive in this sector . A better understanding of precisely which aspects contribute to this advantage is required. This thesis aims to examine the factors that influence the level of technical efficiency at farm level. If such factors could be accurately identified and rectified, farming efficiency, and thus profitability, could potentially be improved. It is hypothesised that TFP will differ based in part on grazing intensity, and specifically that grazing intensity has positively contributed to Ireland's relative competitiveness in the EU.

	\begin{itemize}[label=]
		\item{\textbf{$H_{2}$: Ireland's competitive standing within the EU is partially attributable to a relatively high usage of grass input.}}
	\end{itemize}

The continued and increasing growth in TFP is crucial to fully capitalising on the nation's competitive advantage.Given that prices for feed concentrates have been volatile and have trended upwards in the recent past, and that the events on the world stage suggest that this trend may continue (e.g. emergence of biofuels, droughts and fire in Australia, a severe winter in the U.S., etc), it is also proposed that an increased proportion of grass in the bovine diet will lead to increased TFP. 

	\begin{itemize}[label=]
		\item{\textbf{$H_{3}$: Increased grass input utilisation will increase TFP}}
	\end{itemize}

This thesis will employ the model originally proposed by \citet{greene05} and later adapted to the Irish agricultural setting by \citet{carroll08} to find the determinants of technical efficiency. Justification for this choice of model can be obtained by consulting \citet{carroll08}, whom also carried out a comparision of several Stochastic Production Frontier (SFP) models in the report. In the portion of their analysis which pertained to the dairy sector they found that this model performs the best in tests for theoretical consistency, and that it does this with fewer restrictive assumptions than other models in the SFP family. With this in mind, this thesis will adopt the their specification and expand on it both by including additional grazing system variables and by applying it to farm-level data from the other EU member states.

\subsection{Data}

Data from the Irish National Farm Survey (NFS) (conducted annually by Teagasc, the Irish Agricultural and Food Authority) is employed. In the survey, each farm animal and hectare of crop is assigned a standard gross margin before being grouped into systems according to the dominant enterprise. Farms are selected so as to attain a representative sample of each system in Ireland. 

While Teagasc has excellent data in the NFS on dairy farm characteristics, it has been limited in its capacity to look at the potential for expansion in that the only data currently available for analysis (1984-2008) is from the period where production was limited by milk quota. At present data from the 1975-1983 is being recaptured from historic magnetic tapes and will be available for analysis in 2011. It is hoped to use this data to look at decisions that were made on dairy farms during the expansion phase prior to 1984. 

Annual Irish agricultural price indexes will also be used to deflate all monetary figures after having been converted to euro where necessary. 

Data from the Farm Accountancy Data Network (FADN) is used for the comparisons to other EU countries. The FADN dataset is an unbalanced panel containing farm-level accountancy data including output, income, costs, investment, and debt across all EU member states. Each EU member state collects its own data in a harmonised framework so that cross-country analyses are facilitated. The NFS is the Irish contribution to the FADN. 

\section{The Model}

If one assumes a Cobb-Douglas production technology then the stochastic frontier is:
\begin{equation}\label{sfront}
\ln{y_{i}}=\beta_{0}+\displaystyle\sum \beta_{k}\ln{x_{ki}}+e_{i}
\end{equation}	
where
\begin{equation}\label{comperr}
 e_{i}=v_{i}-u_{i}.
\end{equation}

The left hand side of \eqref{sfront} is the natural logarithm of the farm's output level $y_{i}$. On the right side  $x_{ki}$ is a vector of $k$ production inputs (capital, labour etc), whilst in \eqref{comperr} the composite error term $e_{i}$ is made up of a statistical noise component $v_{i}$ and a non-negative technical inefficiency component $u_{i}$. One adds the time subscript $t$ to apply the model to a panel data setting.

One may assume that technical inefficiency is time-invariant in a short panel data setting (i.e. eliminate the $t$ subscript for the inefficiency term), but this becomes harder to justify as the time horizon of the analysis gets larger, and even more so when the data is an unbalanced panel. This study will utilise an unbalanced panel data set which spans over 30 years, and so the working assumption is that the inefficiency term varies over time.

The True Fixed Effects (TFE) and True Random Effects (TRE) models  are attributable to \citet{greene05}. They use an inefficiency term which is separated into time-invariant heterogeneity and  time-varying inefficiency. The unobserved heterogeneity (or possibly time-invariant inefficiency) is modelled directly in the production function using farm specific dummy variables and estimated in a one-step maximum likelihood approach;
\begin{equation}\label{TFE}
\ln{y_{it}}=\alpha_{i}+\displaystyle\sum\limits_{k=1}^K \beta_{k}\ln{x_{kit}}+v_{it}-u_{it},
\end{equation}	
where $\alpha_{i}$ are farm-specific, time-invariant dummy variables and the inefficiency term is a time-varying random variable. 

To get the TRE model is then only a matter of adding a random farm-specific time-invariant constant term to the TFE model;
\begin{equation}\label{TRE}
\ln{y_{it}}=(\beta_{0}+w_{i})+\displaystyle\sum\limits_{k=1}^K \beta_{k}\ln{x_{kit}}+v_{it}-u_{it},
\end{equation}	
where  $w_{i}$ is a time-invariant, farm-specific random term again intended to capture cross- farm time-invariant heterogeneity. Both models assume that the error term and the inefficiency term are independently and identically distributed normal and halfnormal respectively.

The usefulness of this approach stems from the idea that farms differ in many aspects that have nothing to do with inefficiency (e.g. soil quality). If these farm specific characteristics do not change over time, then the time-invariant term in the TRE model can capture and control for this heterogeneity, making the inefficiency identifiable. This specification is useful and appropriate to the extent that TRE separates heterogeneity from inefficiency, but it is possible that the time-invariant terms may also capture some portion of farm inefficiency if it too is constant over time. This would downwardly bias estimates of farm inefficiency. The literature is still undecided on this point, and so one may also see the time-invariant term interpreted as time-invariant inefficiency. 

As a final step to get to the model which will be employed in this thesis, a translog production technology is assumed with annual time dummy variables to capture neutral technical change. This yields
\begin{equation}\label{carroll}
	\begin{split}
		\ln{y_{it}}= & (\beta_{0}+w_{i})+\displaystyle\sum\limits_{k=1}^K\beta_{k}\ln{x_{kit}}+0.5\displaystyle\sum\limits_{k=1}^K\displaystyle\sum\limits_{j=1}^K\beta_{kj}\ln{x_{kit}}\ln{x_{nit}}\\
		&+{}\displaystyle\sum\limits_{t=1}^T\delta_{t}D_{t}+ v_{it}+u_{it},
	\end{split}
\end{equation}
where  $D_{t}$ are annual dummy variables. These will provide some measure of control for once-off events such as a bumper crop due to favourable weather conditions in a given year for example.

The estimated parameters and inefficiency estimates are used to construct a generalised Malmquist productivity index;
\begin{align}
	TC_{s,t}	&=	\exp{(\delta_{t}-\delta_{s})} \label{TC},	\\
	TEC_{s,t}	&=	\frac{E(\exp{(-u_{it})}\mid e_{it})}{E(\exp{-u_{is}}\mid{e_{is}})} \label{TEC},
\end{align}
and

\begin{equation}\label{SEC}
      SEC_{s,t}=\exp{\left(0.5\displaystyle\sum\limits_{n=1}^N\left(\varepsilon_{nis}SF_{is}+\varepsilon_{nit}SF_{it}\right)\ln{\left(\frac{x_{nit}}{x_{nis}}\right)}\right)}
\end{equation}
where 
\begin{align}
	SF_{is}				&=\frac{\varepsilon_{is}-1}{\varepsilon_{is}} \label{SFis},\\[10pt]
	\varepsilon_{is}	&=\displaystyle\sum\limits_{n=1}^N \varepsilon_{nis} \label{retsc}, 
\end{align}	
and
\begin{equation}
	\varepsilon_{nis} 	=\frac{\delta \ln{q_{is}}}{\delta \ln{x_{nis}}} \label{ielast}.
\end{equation}

The index is based on the approach outlined by \citet{coelli05} where TFP change from year $s$ to $t$ is the product of technical change (TC)  \eqref{TC}, technical efficiency change (TEC) \eqref{TEC} and scale efficiency change (SEC) \eqref{SEC}. The calculation of technical change follows that of \citet{cuesta00}, and is calculated as the difference in the parameters of the time dummy variables in years $s$ and $t$.



%\ssp
\bibliographystyle{agsm}
\bibliography{pgbib}

\end{document}
%%%